%%%%%%%%%%%%%%%%%%%%%%%%%%%%%%%%%%%%%%%%%%%%%%%%%%%%%%%%%%%%%%%%%%%%%%%%%%%%%%%%%%
%% This project aims to create a test template or exercise list from the        %%     
%% Federal University of Ceará (UFC).                                           %%
%% author: Maurício Moreira Neto - Doctoral student in Computer Science         %%
%% contacts:                                                                    %%
%%    e-mail: maumneto@ufc.br                                                   %%
%%    linktree:  https://linktr.ee/maumneto                                     %%
%%%%%%%%%%%%%%%%%%%%%%%%%%%%%%%%%%%%%%%%%%%%%%%%%%%%%%%%%%%%%%%%%%%%%%%%%%%%%%%%%%
\documentclass{ufcatemplates}

\usepackage[utf8]{inputenc}
\usepackage[portuguese]{babel}

%% Informations that will be insert in the table header 
\def\course{Nome da Disciplina}
\def\prof{Nome do Autor}
\def\semester{XXXX.X}
\def\codeCourse{XXXXXX}
\def\registration{}
\def\student{}
\def\graduate{Nome do Curso}
\def\theme{Descrição do Tema}

\begin{document}
    %% Table with the header
    \makeheaderexam

    %% Space for the instructions
    \fbox{
        \parbox{\textwidth}{
            \begin{minipage}{\textwidth}
                \makeinstructions
                {
                    \begin{instlist}
                        \item A avaliação é individual e não é pesquisada.
                        \item Preencha o cabeçalho da folha pergunta com seus dados.
                        \item  Todas as folhas respostas devem conter o nome a a matrícula do aluno.
                        \item O preenchimento das respostas deve ser feito utilizando caneta (preta ou azul).
                    \end{instlist}
                }
            \end{minipage}
        }
    }
    % Space between the instructions and the questions.
    \vspace{1cm}
    

    \question As questões podem ser elencadas usando os comandos \textit{begin} e \textit{end} com o argumento question. Desta forma, cada questão é referente a um \textit{item} da estrutura \textit{question}.\points{2}

    \begin{sol}
      A solução da questão
      \lstinputlisting[language=Python, caption={Python 3 implementing figure left wheel.}, label=cpr]{codes/prog.py}
    
    \end{sol}

    \question Outra funcionalidade interessante: caso você queira colocar pontuação em cada questão, basta colocar o comando \textit{points} com o valor da pontuação da questão. \points{3.5}
        
    \question A ideia é que este template seja sempre atualizado, visando atender a todas as necessidades dos usuários da UFC. Caso possua alguma dica que melhore o template, é possível entrar em contato através do e-mail disposto nos comentários.\points{4}

\end{document}