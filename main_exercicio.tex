%%%%%%%%%%%%%%%%%%%%%%%%%%%%%%%%%%%%%%%%%%%%%%%%%%%%%%%%%%%%%%%%%%%%%%%%%%%%%%%%%%
%% This project aims to create a test template or exercise list from the        %%     
%% Federal University of Ceará (UFC).                                           %%
%% author: Maurício Moreira Neto - Doctoral student in Computer Science         %%
%% contacts:                                                                    %%
%%    e-mail: maumneto@ufc.br                                                   %%
%%    linktree:  https://linktr.ee/maumneto                                     %%
%%%%%%%%%%%%%%%%%%%%%%%%%%%%%%%%%%%%%%%%%%%%%%%%%%%%%%%%%%%%%%%%%%%%%%%%%%%%%%%%%%
\documentclass{ufcatemplates}

\usepackage[utf8]{inputenc}
\usepackage[portuguese]{babel}

%% Informations that will be insert in the table header 
\def\course{Nome da Disciplina}
\def\prof{Nome do Autor}
\def\semester{XXXX.X}
\def\codeCourse{XXXXXX}
\def\registration{}
\def\student{}
\def\date{08/09/2025}
\def\graduate{Nome do Curso}
\def\theme{Descrição do Tema}

\begin{document}
    %% Table with the header
    \makeheaderexercise

    % Space between the instructions and the questions.
    \vspace{1cm}

    \begin{enumerate}
        \item As questões podem ser elencadas dentro do ambiente \verb|enumerate|.
        Desta forma, cada questão é referente a um \textit{item} da estrutura.\points{2}
        \item No final de cada questão a função \verb|\points{n}| pode ser usada para 
        indicar a pontuação da questão. \points{0.5}
    
        \item O código fonte pode ser inserido utilizando o pacote \verb|listings|.
        Veja o exemplo abaixo. \points{2}
        \lstinputlisting[language=Python, label=cpr]{codes/prog.py}
    \end{enumerate}

\end{document}